\documentstyle[hyperref, margin, line]{res}


\hypersetup{backref, %pdfpagemode=none,%FullScreen,
colorlinks=true,backref, urlcolor=blue, %filecolor=blue
}

 \oddsidemargin -.6in
 \evensidemargin -.6in

% \topmargin 2cm
 \addtolength{\topmargin}{-.5in}
 \addtolength{\textheight}{1in}

\textwidth=6.0in \itemsep=0in
\parsep=0in

\newenvironment{list1}{
  \begin{list}{\ding{113}}{%
      \setlength{\itemsep}{0in}
      \setlength{\parsep}{0in} \setlength{\parskip}{0in}
      \setlength{\topsep}{0in} \setlength{\partopsep}{0in}
      \setlength{\leftmargin}{0.17in}}}{\end{list}}
\newenvironment{list2}{
  \begin{list}{$\bullet$}{%
      \setlength{\itemsep}{0in}
      \setlength{\parsep}{0in} \setlength{\parskip}{0in}
      \setlength{\topsep}{0in} \setlength{\partopsep}{0in}
      \setlength{\leftmargin}{0.2in}}}{\end{list}}


\renewcommand{\namefont}{\Huge\textsf}

\begin{document}

\sffamily

 \name{Jithu Sunny \vspace*{.1in}} \hfill March 2012

\begin{resume}


\section{\sc {\bf \textsf{CONTACT INFORMATION}}}
\vspace{.05in}
\begin{tabular}{@{}p{2in}p{4in}}
Kalapurakkal House   & {\it Mobile:}  +91 9567 175 660 \\
Madonna Nagar   & {\it Resi:}    +91 4872 357 392   \\
P.O. Ollur & {\it E-mail:}
\href{mailto:jithusunnyk@gmail.com}{\underline{jithusunnyk@gmail.com}}
\\
Thrissur, Kerala  & {\it WWW:}
\href{http://jithusunnyk.blogspot.com}{\underline{http://jithusunnyk.blogspot.com}}
 \\
India
\end{tabular}

\vspace*{+3.5mm}
\section{\sc {\bf \textsf{BIOGRAPHICAL \qquad DATA}}}
Age: 21\\
Gender: Male\\
Citizenship: India\\
Date and place of birth: 28th September 1990, Thrissur, Kerala \\

\vspace*{+3.5mm}

\section{\bf \textsf{OBJECTIVE}}
To pursue a career through which I could continue to explore, learn \& put in action the technology for a purpose.

\vspace*{+3.5mm}

\section{\bf \textsf{EDUCATION}}
{\bf \textsf{Government Engineering College}}, Thrissur, Kerala.\\
\vspace*{-.1in}

\begin{list1}
\item[] B.Tech in Computer Science \& Technology, June 2008 - March 2012(pursuing)  \qquad (72.5\%)\\ University of Calicut 
\end{list1}

{\bf \textsf{Don Bosco HSS}}, Mannuthy, Kerala.\\
\vspace*{-.1in}
\begin{list1}
\item[] Plus-Two, 2006-2008 - Kerala State Higher Secondary Education\qquad \qquad(94.16\%)
\end{list1}

{\bf \textsf{Holy Angel's HS}}, Ollur, Kerala.\\
\vspace*{-.1in}
\begin{list1}
\item[] SSLC, 2005-2006 - Kerala State Board\qquad \qquad \qquad(88\%)
\end{list1}

\vspace*{+3.5mm} 

\section{\sc \bf \textsf{B.Tech MAIN PROJECT}}
{\bf \textsf{Webtop}}\\
\vspace*{-.1in}

\begin{list1}
\item[] The aim of this innovative project is to create a flexible and efficient, ubiquitous desktop that gives the user power to access his desktop from anywhere. \qquad (In progress)
\end{list1}

\vspace*{+3.5mm}

\section{\sc \bf \textsf{B.Tech \qquad SEMINAR}}
{\bf \textsf{Augmented Reality: Technology \& Applications}}\\
\vspace*{-.1in}

\begin{list1}
\item[] The purpose of this topic is to get acquainted with Augmented Reality which enhances the level of perception and is often termed as a revolutionary user interface and interaction technique. The presentation covers the basic computer vision algorithms, hardware technologies and real-world applications of Augmented Reality, etc.
\end{list1}

\vspace*{+3.5mm}

\section{\sc \bf \textsf{B.Tech MINI PROJECT}}
{\bf \textsf{Python controlled General Purpose Vehicle}}\\
\vspace*{-.1in}

\begin{list1}
\item[] In this Embedded Systems project, we created the prototype of a vehicle that can be controlled using a remote computer. Through the GUI developed using PyQt4 the car can be controlled. The little intelligence in the vehicle is an Atmel Atmega8 microcontroller which is programmed in C and compiled using gcc-avr cross compiler. RF modules are used for wireless communication
\\
\item[] \textbf{Source:} \href{https://github.com/jithusunny/Mini-Project}{\underline{https://github.com/jithusunny/Mini-Project}}
\end{list1}


\vspace*{+3.5mm}
 \section{\sc \bf \textsf{TECHNICAL SKILLS}}
\textbf{Languages \& APIs}
\begin{itemize}
\item \textbf{Proficient:} C, Python, Lisp, GUI Programming - PyQt4, Microcontroller Programming(In C for AVR family), HTML.
\item \textbf{Familiar:} Java, BASH scripting, 8086 Assembly language, Prolog.
\item \textbf{Pursuing:} PHP, Javascript, CSS.
\end{itemize}

\textbf{Source Code Version Control Systems}
\begin{itemize}
\item Git - \href{https://github.com/jithusunny}{\underline{https://github.com/jithusunny}}
\end{itemize}

\textbf{Platforms Worked}
\begin{itemize}
\item GNU/Linux, Microsoft Windows.
\end{itemize}

\textbf{Authoring \& Graphics}
\begin{itemize}
\item \LaTeX\ , \LaTeX\ Beamer, GIMP
\end{itemize}


\vspace*{+2mm}

\section{\sc \bf \textsf{AS A SPEAKER}}
February 2010 \textbf{Venue} : FreeZone, Govt. Engineering College, Thrissur. \textbf{Topic} : Stellarium

\vspace*{+2mm}

\section{\sc \bf \textsf{ACHIEVEMENTS, ACTIVITIES \& HOBBIES}}
\begin{itemize}
\item Winner of \textbf{Mobme Codejam, A 14-day State wide Programming Competition} organized in June 2011 by Mobme Wireless, on two consecutive days - \href{http://codejam.mobme.in/results}{\underline{http://codejam.mobme.in/results}}
\item First prize in \textbf{Online C-Programming competition} hosted by Vidya Academy of Science and Technology in relation with Perizia 3.0 in April 2011.
\item First Prize in \textbf{Ciphering} \& \textbf{Coding} competitions held at Invento'12 - Govt Engg College, Palakkad in March 2012.
\item Developed \textbf{The Resultmanager} program for assisting the teachers to fetch and analyse the results of any batch of students under Calicut University. The program is written in Python. It does the following:
\begin{itemize}
\item Fetch the result pages of a specified batch of students. 
\item Calculate the individual total marks and percentage.
\item Find out the class topper.
\item Calcutate the class-wise \& subject-wise pass percentage.
\end{itemize}

\textbf{Code Repo:} \href{https://github.com/jithusunny/Resultmanager}{\underline{https://github.com/jithusunny/Resultmanager}}

\item Secured second prize in a quiz competition jointly organized by Microsoft \& IEEE at Govt. Engineering college in Jan 2011.
\item Participate regularly in technical events. Have won prizes in competitions like Programming, Quiz, etc. in many Technical festivals.
\item Active \textbf{Free Software Hacktivist} and member of Free Software User's Group(FSUG) Thrissur.
\item Love coding in Python, admire Functional Programming, typeset using \LaTeX\ \& support UNIX philosophy.
\item \textbf{Electronics hobbyist}. Spends time hands on Embedded systems and microcontroller programming.
\item \textbf{Singer}, bagged prizes in several competitions at school level and sub-district level. Appreciate seemingly unmixable genres of music like Classical \& Metal.
\item A \textbf{Tabala} player.
\item In love with \textbf{deep forests} and wild nature.
\item Favorite pastime after Trekking is reading. A budding up fiction writer.
\end{itemize}

\vspace*{+2mm}

\section{\sc \bf \textsf{BLOG}}
I maintain a blog at \href{http://jithusunnyk.blogspot.com}{\underline{http://jithusunnyk.blogspot.com}}, powered by Blogger.

\section{\sc {\bf \textsf{REFERENCES}}}
\vspace{.05in}
\begin{tabular}{@{}p{2in}p{4in}}
Valsaraj K S   &  Helen K J \\
Assistant Professor \& HOD   & Assistant Professor   \\
GEC, TCR   &  GEC, TCR  \\
0487-26809230 & 9446352699 \\
\href{mailto:valsaraj.ks@gmail.com}{\underline{valsaraj.ks@gmail.com}} & 
\href{mailto:helenkj28@gmail.com}{\underline{helenkj28@gmail.com}}
\end{tabular}

\end{resume}
\end{document}
